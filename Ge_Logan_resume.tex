\documentclass{resume} % Use the custom resume.cls style

\usepackage[left=0.4in,top=0.4in,right=0.40in,bottom=0.3in]{geometry} % Document margins
\usepackage{hyperref}

%Document line spacing
\usepackage{setspace}

\newcommand{\tab}[1]{\hspace{.2667\textwidth}\rlap{#1}} 
\newcommand{\itab}[1]{\hspace{0em}\rlap{#1}}

\name{Logan Ge} % Your name
% You can merge both of these into a single line, if you do not have a website.
\address{+1(312) 826-9860 \\ Chicago, IL \\ gelogan@uchicago.edu \\ \href{http://github.com/geloganu}{github.com/geloganu}}

\begin{document}

%----------------------------------------------------------------------------------------
%	EDUCATION SECTION
%----------------------------------------------------------------------------------------
\vspace{-0.6em}
\begin{rSection}{Education}
{\bf University of Chicago}\hfill {Expected Jun, 2023}\\
B.A. in Physics, Honors $|$ GPA 3.74/4.00
\smallskip\\
\textbf{Relevant Coursework}: Solid State Physics, Adv. Quantum Mechanics, Statistical \& Thermal Physics, General Relativity \& Cosmology, Computational Physics, Machine Learning, Experimental Physics, Statistical Models \& Methods, Abstract Linear Algebra, Quantitative Portfolio Management \& Algorithmic Trading.
\smallskip\\
\textbf{Honors \& Awards}: Honors Degree, James Franck Institute Research Fellowship, Metcalf Fellowship.


\end{rSection}

%----------------------------------------------------------------------------------------
% Work experience
%----------------------------------------------------------------------------------------
\vspace{-0.5em}
\begin{rSection}{EXPERIENCE}

\textbf{Kang Group at the University of Chicago} \hfill October 2021 - Present\\
\textit{Research Assistant} \hfill \textit{Chicago, IL}
\vspace{-0.6em}
 \begin{itemize}
   \itemsep -5.8pt {}
   \item Developed and deployed scientific simulation program in Python to model quantum Hall condensed matter systems, utilizing exact diagonalization and numerical linear algebra algorithms with Numpy and Scipy.
   \item Integrated multicore parallel computing capabilities through Linux bash scripts, increasing simulation speed by 150\% and enabling the analysis of larger data sets.
   \item \textbf{Notable Projects}\begin{itemize}
      \item \vspace{-0.6em} \textbf{ANN Disordered Phase Transition}\hfill Apr 2023
      \begin{itemize}
         \vspace{-0.6em}
         \itemsep -5.8pt {}
         \item Preprocessed and initialized model dataset for disordered pfaffian state in condensed matter physics.
         \item Training artificial neural network using TensorFlow to predict properties and phase transitions of disordered fractional quantum Hall states for highly accurate experimental conditions.
      \end{itemize}
      \item \vspace{-0.6em}\textbf{Thesis}: \textit{Numerical study of the $\nu=5/2$ state and favourability of the Moore-Read wave function}
      \begin{itemize}
         \vspace{-0.6em}
         \itemsep -5.8pt {}
         \item Spearheaded independent honors thesis project conducting computational numerical research of the effectiveness of the Pfaffian class wave functions for the $\nu=5/2$ fractional quantum Hall effect.
         \item Presented projects and numerical results to leading experts and University of Chicago faculty.
          
      \end{itemize}
   
    
\end{itemize}

 \end{itemize}
 
\textbf{James Franck Institute} \hfill June 2022  - September 2022 \\
\textit{Summer Undergraduate Researcher} \hfill \textit{Chicago, IL}
\vspace{-0.6em}
 \begin{itemize}
   \itemsep -5.8pt {}
   \item Implemented numerical modelling pipeline that consolidated data collection/generation, theoretical modelling, and information visualization, reducing time spent coding by lab members and increasing lab research efficiency.
   \item Configured efficient experimental layout and improved ultra-high vacuum efficiency in collaboration with faculty.
    
 \end{itemize}

\end{rSection} 

%----------------------------------------------------------------------------------------
%Project section
%----------------------------------------------------------------------------------------
\vspace{-0.5em}
\begin{rSection}{PROJECTS}
\vspace{-1.25em}

\item \textbf{Tomography Data Processing and Image Reconstruction} \hfill April 2022 - May 2022
\begin{itemize}
   \vspace{-0.4em}
   \itemsep -5.8pt {}
   \item Designed and built discrete particle data collection trials for two dimension radioactive tomography imaging.
   \item Integrated statistical and data fitting models that pinpointed feature locations with 500\% higher accuracy for raw data reconstruction.
   
\end{itemize}

\item \textbf{Mean Variance Portfolio Management} \hfill August 2021
\begin{itemize}
   \vspace{-0.4em}
   \itemsep -5.8pt {}
   \item Led team of 4 students to develop a multifunctional portfolio management tool and analyze portfolio performance.
   \item Conducted portfolio analysis for risk management and asset allocation strategies using multivariate regression analysis and forecasting with data visualization.
   
\end{itemize}


\end{rSection} 


%----------------------------------------------------------------------------------------
%Leadership
%----------------------------------------------------------------------------------------
\vspace{-0.5em}
\begin{rSection}{Leadership and PROFESSIONAL DEVELOPMENT} 
\vspace{-1.25em}
\item \textbf{Boston Consulting Group} \hfill March 2023 \\
\textit{Data Science \& Advanced Analytics Virtual Experience Participant at The Forage}
   \begin{itemize}
   \vspace{-0.6em}
   \itemsep -5.8pt {}
      \item Identified key business insights for client needs by performing EDA and feature engineering in BCG's simulated project.
      \item Delivered forecast analysis for customer churn savings using random forest models with scikit-learn, achieved 91\% target accuracy in identifying potential churned users.

   \end{itemize}

\textbf{The Campanile Project} \hfill July 2020 - July 2021 \\
\textit{Board Member}
 \begin{itemize}
   \vspace{-0.6em}
   \itemsep -5.8pt {}
    \item Coordinated collaborative effort between six students to create volunteer platform resulting in increased accessibility of program outreach by ten folds at 75\% cost savings compared to the voluntourism industry.
    \item Installed database system to house over 500 volunteer opportunity listings, improved efficiency by 300\%.
    
 \end{itemize}

\end{rSection}
%----------------------------------------------------------------------------------------
%Skills Section
%----------------------------------------------------------------------------------------
\vspace{-0.5em}
\begin{rSection}{SKILLS}

    \begin{tabular}{ @{} >{\bfseries}l @{\hspace{6ex}} l }
    Programming Languages  & Python, R Studios, Julia, Fortran, Linux shell scripting, COMSOL   \\

    Technologies  &   SQL, TensorFlow, Scikit-Learn, SciPy, Pandas/NumPy,  Matplotlib, Git\\

    Interests & Electric Guitar, Skiing, F1, Chicago Bulls
    \end{tabular}\\
    \end{rSection}


\end{document}